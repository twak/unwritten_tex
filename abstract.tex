\vspace*{1.75in}
\begin{center} {\bf Abstract}\end{center}

Creating virtual models of urban environments is essential to a disparate range of applications, from geographic information systems to video games. However, the large scale of these environments ensures that manual modeling is an expensive option. Procedural modeling is a automatic alternative that is able to create large cityscapes rapidly, by specifying algorithms that generate streets and buildings. Existing procedural modeling systems rely heavily on programming or scripting --- skills which many potential users do not possess. We therefore introduce novel user interface and geometric approaches, particularly generalisations of the straight skeleton, to allow urban procedural modeling without programming.

We develop the theory behind the types of degeneracy in the straight skeleton, and introduce a new geometric building block, the mixed weighted straight skeleton. In addition we introduce a simplification of the skeleton event, the generalised intersection event. We demonstrate that these skeletons can be applied to two urban procedural modeling systems that do not require the user to write programs.

The first application of the skeleton is to the subdivision of city blocks into parcels. We demonstrate how the skeleton can be used to create highly realistic city block subdivisions. The results are shown to be realistic for several measures when compared against the ground truth over several large data sets. 

The second application of the skeleton is the generation of building's mass models. Inspired by architect's use of plan and elevation drawings, we introduce a system that takes a floor plan and set of elevations and extrudes a solid architectural model. We evaluate the interactive and procedural elements of the user interface separately, finding that the system is able to procedurally generate large urban landscapes robustly, as well as model a wide variety of detailed structures. 
